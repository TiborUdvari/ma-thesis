% Language
\specialsection{Language}{}{black}{white}
\subsection{Make it simple}
\subsection{Removing complexity}
\subsection{The evolution}

\subsection{Software Literacy: Reading and Writing}

% Contextualizing the Open Source Movement
The open source movement has undeniably influenced a myriad of practical applications, ranging from operating systems like Linux to various software tools. However, its impact on the realm of creative software tools has been comparatively limited. According to Reas and Fry~\parencite[30]{reasProcessingProgrammingHandbook2007}, companies like Adobe and Microsoft largely dominate the field, limiting the open-source philosophy's penetration into the culture of arts software. % Here, you establish the broader impact of open source and point out a gap in its influence over artistic software.

% Importance of Open Source Ethos in Processing
Within this context, Processing emerges as a significant counterexample, embedding the ethos of open source at its very core. This ethos is not simply a byproduct but an intentional design choice to promote software literacy. Reas and Fry define literacy in the context of software as the ability to both "read" and "write" within a medium~\parencite[29]{reasProcessingProgrammingHandbook2007}. % This highlights how Processing intentionally incorporates open source values to facilitate a specific kind of literacy.

% Elaborating Software Literacy
In traditional print writing, literacy involves generating rhetorical tools that aim to demonstrate and convince. In contrast, computer-based literacy enables one to create processes that can simulate and decide. % This part elaborates on what software literacy entails, contrasting it with traditional forms of literacy.

% Processing's Alignment with Design Principles
Moreover, Processing exemplifies many of the Design Principles for Tools to Support Creative Thinking as suggested by Resnick~\parencite{resnickDesignPrinciplesTools}. Particularly, by fostering a culture of collaboration and open exchange, Processing does not merely align with open source motivations but elevates itself as a superior Creativity Support Tool (CST). % This wraps up the section by connecting Processing's open-source ethos to broader design principles, emphasizing its role as a potent CST.