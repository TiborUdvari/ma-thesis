\section{Integrate}

\begin{figure}[h] 
    \centering 
    \includegraphics[width=1\textwidth]{} 
    \caption[Ben Fry at PCD 2018]{Ben Fry with conference attendees. Source: \citefield{guptaBenFryConference2018}{author}, Medium, \citeyear{guptaBenFryConference2018}.}
    \label{figure:benfry-pcd}  
  \end{figure}
  \textbf{Note.} Source: \textcite{guptaBenFryConference2018}.
  
  \subsection{The ingredients of the Processing Project}

  Ben Fry shed light on the holistic nature of the Processing initiative, noting:
  \begin{quote}
  "The processing project is a community, a piece of software that you run, and a language. And that order is important." – Ben Fry \parencite[19:22]{artsatmit2017CASTSymposium2017}
  \end{quote}
  
  In its design philosophy, Processing introduced the concept of "software sketches". It was designed with an accessible entry point for beginners, while also providing advanced capabilities for experienced users \parencite{reasProcessingProgrammingMedia2006}.
  
  Over the years, Processing has been adopted across various disciplines, showcasing its versatility. To further its impact and development, the Processing Foundation was established, with notable contributors like Daniel Shiffman. The foundation aims to support and expand the reach of the Processing software and its associated projects.
    
  
  
  % todo NN ne pas distinguer la litt review de l’introduction, ni la méthodo, il faudrait tout mettre dedans pour montrer (1) en quoi cela permet de formuler une question de recherche (qui reste à expliciter ici), (2) mettre en place une méthodologie pour y répondre
  
  % methodologie
  
  % todo NN Rappelle toi qu’une revue de littérature sert à préciser ton questionnement général décrcit en introduction et formuler une question de recherche puis une méthodo pour y répondre. Tu le fais heureusement déjà un peu !
  
  % todo NN et en fin de cette grosses introduction tu pourrais mettre l’annonce du plan du mémoire, ce que vont contenir les chapitres
  
  % \section{Literature review}
  
  