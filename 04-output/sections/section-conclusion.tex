\specialsection{Conclusion}{}{black}{white}

* The community as well as the software plays an important role
* It is difficult to get people involved in the long term, most people that contributed code, was either for a personal project, often in the form of studies. Others created tools for theirselves. They took the time during their studies.
* The online community, in the form of the alpha forum helped users from regions such as the Philippines or Australia to get a community and get involved
* This allowed people from everywhere to learn, even by reading conversations
* While it was very friendly for beginners, more advanced people expressed frustrations
* There was an element of the right thing at the right time. The technology of allowing rapid distribution through java applets, the democratisation of computers and the internet allowed it to spread and have a distributed community
* People seemed to value it as a tool , the best tool at the time for them as opposed to something
* People who wanted to explore computational design
* The community was an important part
* Open source philosophies or personal convictions didn't play an important part
* Processing enabled a lot of beginners to get into coding
* The prevalence of it being used in education made it the default
* The sustainability of the project software wise is in question
* The API is very stable, there are not a lot of changes to it, the advantage is that we can run things from very far back with little modification.
* According to some interviewees, this is also a drawback because it can hinder exploration into other areas
* The fact of having everyone on the same forum created a sense of community that everybody knew each other.
* After it exploded with some scale the incentives changed and it was difficult to maintain
* Because it was easier for Ben to just fix bugs in the short term instead of onboarding people, this was not good, something else could have been better
* We should be thankful for people who maintain these tools so that we can still use them
* The excitement over processing seems a little less hard than for tools like blender or even touch designer, who have different paradigms
* The fact that it seems to make it conversly harder to get contributions, maybe after it has achieved this level it should have a different model for handling like signing issues etc.
* There is a culture of promoting in every way we can, there should probably be a culture of somehow finding ways to give back to the tool itself. A lot of the times they talk about making it so people can build their own photoshop, the learning opportunity for building for a lot of people would probably be better.


* Suggestion
