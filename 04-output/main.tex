\documentclass{memoir}
\usepackage[a4paper,top=2cm,bottom=2cm,left=3cm,right=3cm,marginparwidth=1.75cm]{geometry}

% Useful packages
\usepackage{amsmath}
\usepackage{graphicx}
\usepackage[colorlinks=true, allcolors=blue]{hyperref}

\title{Charting the Beginnings of the Processing Community: Implications for Collaborative Open Source Art Development}
\author{Tibor Udvari}

\begin{document}

\maketitle

\begin{abstract}
What are the dynamics that lead to the creation of a community around a piece of software? How do these dynamics influence the development of the software itself? This thesis explores these questions through the lens of the Processing community, a community of artists and designers who use the Processing software to create interactive art. The paper traces the history of the Processing community from its beginnings in 2001 to the present day, using a combination of interviews, archival research, and analysis of the software development cycle itself. The paper finds that the Processing community was created through a combination of factors, including the software's design, the community's shared values, and the community's shared history. The paper also finds that the community's shared history has influenced the development of the software itself, with the community's values and history influencing the software's design. The paper concludes by discussing the implications of these findings for the development of collaborative open source art software.
\end{abstract}

%\tableofcontents

\section{Introduction}

% introduce context
% -- about open source communities in general
% -- why study the processing community in particular
% -- the particularities of open source + creative coding, if there was such a thing
% -- importance of open source projects, maybe mention how flash died out, and the recent unity price change ?

\end{document}