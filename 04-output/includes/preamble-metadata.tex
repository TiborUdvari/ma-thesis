\title{Code, Community, and Creativity: Navigating the Complexities of Collective Contribution and Sustainability in the Processing Project}
\author{Tibor Udvari}

\newcommand{\myTitle}{Code, Community, and Creativity: Navigating the Complexities of Collective Contribution and Sustainability in the Processing Project}
\newcommand{\myAuthor}{Tibor Udvari}
\newcommand{\myShortTitle}{
    Code, Community, and Creativity in Processing
}
\newcommand{\myAbstract}{
    What are the dynamics that lead to the creation of a community around a piece of software? How do these dynamics influence the development of the software itself? This thesis explores these questions through the lens of the Processing community, a community of artists and designers who use the Processing software to create interactive art. The paper traces the history of the Processing community from its beginnings in 2001 to the present day, using a combination of interviews, archival research, and analysis of the software development cycle itself. The paper finds that the Processing community was created through a combination of factors, including the software's design, the community's shared values, and the community's shared history. The paper also finds that the community's shared history has influenced the development of the software itself, with the community's values and history influencing the software's design. The paper concludes by discussing the implications of these findings for the development of collaborative open source art software. 
}
\newcommand{\myInstitution}{Media Design Department,\\Geneva School of Art and Design (HEAD – Genève),\\University of Applied Sciences and Arts Western Switzerland (HES-SO)}

\newcommand{\myAdvisorShort}{Prof. Nicolas Nova, Ph.D.}

\newcommand{\myAdvisorMedium}{Dr. Nicolas Nova, Professor at HEAD – Genève, holds dual PhDs in Social Sciences and Human-Computer Interaction from the University of Geneva and EPFL, respectively. He is a co-founder of the Near Future Laboratory and specializes in digital anthropology and design research.}

\newcommand{\myAdvisorLong}{Dr. Nicolas Nova is a distinguished Professor at the Geneva School of Art and Design (HEAD – Genève), where he teaches digital anthropology, ethnography, and design research. As a co-founder of the Near Future Laboratory, he is at the forefront of exploring the interplay between digital culture and technology from a socio-anthropological perspective. His academic prowess is backed by dual doctorates: a PhD in Social Sciences from the University of Geneva and another in Computer Science with a focus on Human-Computer Interaction from the Swiss Federal Institute of Technology (EPFL). His previous academic engagements include positions as a visiting professor at Politecnico di Milano and a visiting researcher at the Art Center College of Design, contributing significantly to the fields of digital media practices and design fiction. Dr. Nova's work is instrumental in shaping new computing experiences that prioritize human needs, cultures, and contexts.}

\newcommand{\myKeywords}{Open Source, Creative Coding, Community Engagement}