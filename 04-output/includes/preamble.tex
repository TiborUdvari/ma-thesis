\documentclass[draft]{article}

\usepackage{lipsum}
\usepackage[utf8]{inputenc}
\usepackage{calc}
\usepackage{fancyhdr}
\usepackage{ragged2e}

\usepackage{amsmath}
\usepackage{graphicx} 
\usepackage{svg}
\graphicspath{{images/}}
\svgsetup{inkscapelatex=false} % - does not use the latex fonts in imported svg

\usepackage{lmodern}
\usepackage{array}  % Allows for more flexible column formatting
\usepackage{booktabs}  % Improves table aesthetics
\usepackage[skip=0.5ex, justification=raggedright, singlelinecheck=false]{caption}
\usepackage{enumitem}
\usepackage{float} % Required for the H specifier
\usepackage{subcaption} % Required for sub-figures

% table
\usepackage{multirow}
\usepackage{tabularx}
\usepackage{hyperref}

\PassOptionsToPackage{colorlinks=true, allcolors=blue}{hyperref}

\usepackage[style=apa, backend=biber]{biblatex}
\addbibresource{main.bib}

% Define Lengths
\newlength{\paperWidth}
\newlength{\paperHeight}
\newlength{\leftMargin}
\newlength{\rightMargin}
\newlength{\topMargin}
\newlength{\bottomMargin}
\newlength{\spiralWidth}

\setlength{\paperWidth}{158mm}
\setlength{\paperHeight}{210mm}
\setlength{\spiralWidth}{10.7mm}

% Define ratios
\newcommand{\marginRatio}{0.15}

% Use the ratio to calculate margins
\setlength{\leftMargin}{\paperWidth*\real{\marginRatio}}
\setlength{\rightMargin}{\paperWidth*\real{\marginRatio}}
\setlength{\topMargin}{\paperHeight*\real{\marginRatio}}
\setlength{\bottomMargin}{\paperHeight*\real{\marginRatio}}

\usepackage[paperwidth=\paperWidth,
    paperheight=\paperHeight, 
    left=\leftMargin, 
    right=\rightMargin, 
    top=\topMargin, 
    bottom=\bottomMargin,
    marginparwidth=0mm,
    marginparsep=0mm,
    headheight=0mm,
    hoffset=\spiralWidth,
    voffset=-5mm,
    headsep=0mm,
    footskip=0mm]{geometry}
\usepackage{xcolor}
\usepackage{titlesec}
\usepackage{afterpage} % for the \afterpage command

% Set up the page style for fancy headers and footers
\pagestyle{fancy}
\fancyhf{} % Clear all header and footer fields

% Remove the header line
\renewcommand{\headrulewidth}{0pt}
% Remove the footer line, if needed
\renewcommand{\footrulewidth}{0pt}

\fancyfoot[R]{\raisebox{-15mm}{\thepage}}


% To move the page number further to the right you can use \fancyfootoffset
%\fancyfootoffset[B]{10mm} % Adjust the offset to move the page number outside the defined text margin

\newcommand{\getyear}[1]{\citeyear{#1}}

\DeclareCiteCommand{\fullcitefigure}
  {\boolfalse{citetracker}%
   \boolfalse{pagetracker}%
   \usebibmacro{prenote}}
  {\printtext{Photo by \printnames{author}, published in \printlist{organization} in \cite*{guptaBenFryConference2018}}%
   \renewcommand{\mkbibparens}[1]{} % Remove parentheses around citation
   \unspace\parencite{\thefield{entrykey}}} 
  {\multicitedelim}
  {\usebibmacro{postnote}}


\DeclareBibliographyDriver{artwork}{%
  \printnames{author}%
  \setunit*{\addspace}%
  \printfield{year}%
  \setunit*{\addspace}%
  \printfield{title}%
  \setunit*{\addperiod\space}%
  \printfield{howpublished}%
  \setunit*{\addperiod\space}%
  \printfield{note}%
  \setunit*{\addperiod\space}%
  \usebibmacro{url+urldate}%
  \finentry
}

% Fonts
\usepackage[]{fontspec}


% Define commands for setting up each font family
\newcommand{\setupInterFonts}{
    \setmainfont{Inter}[
        Path = ./fonts/Inter/,
        Scale = MatchLowercase,
        UprightFont = *-Regular,
        BoldFont = *-Bold,
        Extension = .ttf
    ]
}

\newcommand{\setupSourceSansFonts}{
    \setsansfont{SourceSans3}[
        Path = ./fonts/SourceSans3/,
        Scale = MatchLowercase,
        UprightFont = *-Regular,
        BoldFont = *-Bold,
        ItalicFont = *-Italic,
        BoldItalicFont = *-BoldItalic,
        Extension = .ttf
    ]
}

\newcommand{\setupSourceSerifFonts}{
    \newfontfamily\seriffont{SourceSerif4}[
        Path = ./fonts/SourceSerif4/,
        Scale = MatchLowercase,
        UprightFont = *-Regular,
        BoldFont = *-Bold,
        ItalicFont = *-Italic,
        BoldItalicFont = *-BoldItalic,
        Extension = .ttf
    ]
}

% Then use the desired command to switch to that font setup
%\setupInterFonts
% \setupSourceSansFonts
% \setupSourceSerifFonts

% Define the command to set up Inter font
%  \newfontfamily\headingfont{Inter}[
%    Path = ./fonts/Inter/,
%    Scale = MatchLowercase,
%    UprightFont = *-Regular,
%    BoldFont = *-Bold,
%    Extension = .ttf
%  ]

%  \newfontfamily\headingfont{SourceSerif4}[
%     Path = ./fonts/SourceSerif4/,
%     Scale = MatchLowercase,
%     UprightFont = *-Regular,
%     BoldFont = *-Bold,
%     ItalicFont = *-Italic,
%     BoldItalicFont = *-BoldItalic,
%     Extension = .ttf
% ]


\newfontfamily\headingfont{SourceSans3}[
    Path = ./fonts/SourceSans3/,
    Scale = MatchLowercase,
    UprightFont = *-Regular,
    BoldFont = *-Bold,
    ItalicFont = *-Italic,
    BoldItalicFont = *-BoldItalic,
    Extension = .ttf
]

\newfontfamily\bodyfont{SourceSerif4}[
    Path = ./fonts/SourceSerif4/,
    Scale = MatchLowercase,
    UprightFont = *-Regular,
    BoldFont = *-Bold,
    ItalicFont = *-Italic,
    BoldItalicFont = *-BoldItalic,
    Extension = .ttf
]

% Apply the body font to the normal text
\renewcommand{\normalsize}{\fontsize{10}{12}\selectfont\bodyfont}
\renewcommand{\large}{\fontsize{12}{14}\selectfont\bodyfont}
\renewcommand{\Large}{\fontsize{14}{18}\selectfont\bodyfont}
% Add any other size commands you need to redefine

% % Apply the Inter font to section headings
\titleformat*{\section}{\Large\headingfont}
\titleformat*{\subsection}{\large\headingfont}
\titleformat*{\subsubsection}{\normalsize\headingfont}