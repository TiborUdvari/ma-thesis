\documentclass{article}
\usepackage[utf8]{inputenc}
\usepackage{geometry}
\usepackage{xcolor}
\usepackage{titlesec}
\usepackage{lipsum} % for dummy text
\usepackage{afterpage} % for the \afterpage command

% Define the geometry for the special page
\newgeometry{left=1in,right=1in,top=1in,bottom=1in}

% Command to start a special page
% #1 for the section name, #2 for the description
\newcommand{\specialsection}[2]{
    \newpage
    \pagecolor{black}\afterpage{\nopagecolor}
    \thispagestyle{empty}
    \begingroup
    \color{white}
    \centering
    \vspace*{\stretch{1}}
    \section*{\Huge\textbf{#1}} % Add the section to the table of contents
    \addcontentsline{toc}{section}{#1} % Manually add the section to the table of contents
    \fontsize{12pt}{14pt}\selectfont
    #2
    \vspace*{\stretch{1}}
    \endgroup
    \newpage
}

\begin{document}

\tableofcontents % Include a table of contents

\clearpage % Start on a new page after the table of contents

% Use the custom command to create a special section page with a title and description
\specialsection{Introduction}{This is the introduction to our document. Here we set the stage for the content that is to follow. \lipsum[1]}

% Reset the geometry and page color for the following pages
\restoregeometry
\nopagecolor

% Continue with the rest of your document
\section{First Section}
\lipsum[2-4]

\end{document}
