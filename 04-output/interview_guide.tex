Interview guide: Ben Fry
% Maybe I should try to categorize the questions
% Motivations
% Perceptions vs reality
% How did you think this would work out, what sense of commitment did you feel to the community?
\begin{itemize}
    \item You have answered so many questions, many of them technical and specific. Did it make you feel like an expert or did it give you some sort of satisfaction?
    \item What do you think would happen to processing if you stopped updating it?
    \item What do you think about Flash and other proprietary technologies.
    \item As seen in the data you are the one who has contributed the most regularly to the project. Other people have mostly come and gone. What do you think about that?
    \item What are your views on how the community developed? Specifially thinking about the contributions to open source software?
    \item What other tools like Flash or Director did you use and why waren't you satisfied with the status quo?
    \item What do you think about the ecosystem today?
    \item It seemed like Processing came at a unique time and place. What did you think were the main factors of it's success? And do you consider it a success today?
    \item Because of Java applets people could try it out easily at home like flash websites, do you think that was a major factor for adoption?
    \item What do you think were the main reasons for the growth / spread of the community? Before Github especially it could be quite difficult for people to discover these sorts of things.
    \item How did p5 happen, the continuation of the thing
    \item How did the early books influence the credibility and the success of the project. I remember it was the era of O'Reilly books being very popular in the literature.
\end{itemize}

Interview guide: Library contributor
\begin{itemize}
    \item What problem were you trying to solve?
    \item How did you get introduced to the processing community, how did you hear about it?
    \item Were you using other tools like Flash or Director and why didn't you continue using those?
\end{itemize}